\chapter{INTRODUCTION}
\thispagestyle{plain}

\label{Introduction}

In this chapter we briefly talk about the current status of information extraction from log files. We also discuss the motivation for our work, the problem that we are trying to solve, and briefly describe the proposed solution. Chapter 2, gives a brief overview of the proposed solution and the concepts involved in this thesis. Chapter 3, briefly touches upon the prior work done in this domain. In the 4th chapter, we elaborately describe the design and architecture of various components in the system.

\section{Background}

The advancement of software has led to creation of huge amounts of data. One such kind of data that is generated behind almost every system today, is logs. Every single thing we do on a software system results in triggering various pieces of software, and in turn creating several lines of logs. Log data is no more just a tool for debugging software. It is more and more used for diagnostics and auditing purposes as it contains a lot of vital information. Understanding the importance of logs, enterprises have started spending a lot of money to generate and store huge amounts of log daily. Also, special emphasis is given to ensure creation of useful log data and to manage it efficiently.

A lot of this data generated in the form of logs is descriptive in nature because that was the primary purpose of logs. Now-a-days, people have started giving emphasis on structured log data. This eases the understanding and management of the logs. Today, a large amount of log files have a structure, which includes the textual descriptions in the end. The structured log files also convey a lot of information with less verbosity. Log formats are also being standardized, so that every vendor follows a format for certain set of tools or softwares. For instance, such recommendations provided in ``Guide to Computer Security Log Management'', published by the National Institute of Standards and Technology (NIST)~\cite{nist_guide}, indicates the growing importance of standardizing and analyzing log files.

\section{Motivation}

