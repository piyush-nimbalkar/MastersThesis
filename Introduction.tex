\chapter{INTRODUCTION}
\thispagestyle{plain}

\label{Introduction}

In this chapter we talk about the current status of information extraction from log files. We also discuss the motivation for our work, the problem that we are trying to solve, and then briefly describe the proposed solution. Chapter 2, gives an overview of the proposed solution and the concepts involved in this thesis. Chapter 3, briefly touches upon the prior work done in this domain. In the 4th chapter, we elaborately describe the design and architecture of the proposed framework. We explain in detail, all the components involved in the system.


\section{Background on Logs}

The advancement of software has led to creation of huge amount of data. One such kind of data that is generated behind almost every system today, is logs. Every single thing we do on a software system results in triggering various pieces of software, and in turn creating several lines of logs. Log data is no longer just a tool for diagnostics and debugging software. It is increasingly used in cybersecurity, forensics, and for auditing purposes, due to the amount of vital information it holds. Recognizing the importance of logs, enterprises have started spending a lot of money to generate and store huge amount of log daily. Also, special emphasis is given to ensure creation of useful log data,  and to manage and analyze it efficiently.

Lot of the data generated in the form of logs is descriptive in nature because that was the primary purpose of logs. Now-a-days, people have started giving emphasis on structured log data. This eases the understanding and management of the logs. Today, a large amount of log files have a structure, which include the textual descriptions at the end. The structured log files also convey more information with less verbosity. Log formats are also being standardized, so that every vendor follows a format for certain set of tools or softwares. For instance, such recommendations provided in ``Guide to Computer Security Log Management'', published by the National Institute of Standards and Technology (NIST)~\cite{nist_guide}, indicate the growing importance of standardizing and analyzing log files.


\section{Motivation}

Log files give a valuable insight into systems and their activities. Looking at the log file, not only can we understand the event, but also trace its origin by audit trails and forensics. For instance, database logs can help trace the modifications on data, while web server log files speak a lot about the resources accessed from the web. On one hand, operating system logs help us figure out what is happening at the system level, on the other, firewall logs help record malicious activities at the network level. Thus, log files form a vital tool in the cybersecurity. Using the information from log files we can pro-actively defend our systems against potential malicious activities and attacks.

It is important to know the structure of the log files to extract more and precise information. There are various difficulties in finding the exact structure of a log file. As mentioned by Kimball and Merz~\cite{kimball_log_problem}, the problems with log file analysis are multiple file formats, incomplete, inconsistent and irrelevant data, and dispersed information. There are even log files which have multi-line log entries.

There have been various attempts to tackle these problems with log files. There are tools that parse specific log files and extract information from them. But this is not a scalable approach as it requires us to know the file format of a log file before analyzing it. Also, there are text processing tools which extract information from log files as they would do from a normal chunk of text. This information is used by Intrusion Detection and Prevention systems to detect malicious behavior. However, if we can leverage the structure of log files, we can clearly understand events in the log files and get more information from them. Often, there are semantic relations between various columns of log files. These may be documented somewhere but it is difficult for an autonomous system to understand the relations as a human would.

Apart from text processing systems, there are enterprise tools like Splunk \footnote{http://www.splunk.com/}, Sumologic \footnote{https://www.sumologic.com/}, etc. These tools primarily focus on log management and analytics. Splunk does detect various fields in the log file but it does not always separate it out in proper columns. This happens particularly when it does not know the structure or source of the log file before hand. Thus, such enterprise tools have some limitations in predicting the structure of unknown log files and further do not deal with finding semantic connections between various columns of the log file.


\section{Contribution}

To solve the said problems we have built a framework, that takes a log file as input and gives out its semantic interpretation as Linked Open Data expressed in RDF (Resource Data Framework)~\cite{brickley2004rdf}. Our framework works for any random log file that has structured columns. We split the log file into identifiable columns and then predict classes for them. Using these columns we generate a list of candidate relations between those columns whose classes we have predicted, which are given out into a RDF file. We have extended the IDSOntology \footnote{http://ebiquity.umbc.edu/ontologies/cybersecurity/ids/v2.3/IDSOntology.owl} to help identify the relationships between various column classes. For this matter the framework is flexible to use any ontology.

The framework was designed such that it is possible to scale it in future. We can add experts to identify different kind of columns as and when needed. We run the framework against log files with different structure and unknown sources. We also test the system against a dataset of randomly generated artificial log files.
