\chapter{Conclusion}
\thispagestyle{plain}

\label{Conclusion}

Repackaged Android malware can change its code and structure in a way that the signatures of the variants are completely different making the detection of the malware variant very difficult. Though there would be significant code changes due change in ‘victim’ repackaged application, in general the behavior of the malware and thus its functionality stays the same. We use the distribution of syscalls to detect a malware and categorize it into the specific known malware family. Our experiments indicate that SdA model performs better than rest of the models for malware detection. While random forest and Adaboost algorithm performs better when we try to classify given application to the specific malware family. We also find that SdA algorithm is robust and stable even with corrupted input data.

As part of future work, it would be interesting to study to the performance of SdA using different stacking algorithms.
