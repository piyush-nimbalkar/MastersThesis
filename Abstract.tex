\newpage
\pagestyle{empty}

\begin{center}
\vspace{0.1in}
\large{\bf ABSTRACT} \par  
\bigskip \bigskip
\end{center}

{\bf Title of Thesis:} \parbox[t]{4.5in}{{Semantic Representation of Log Files\\
Piyush Nimbalkar, MS Computer Science, 2015\\}}
\begin{singlespace}
{\bf Thesis Directed by:}{\hspace{2.5mm}} \parbox[t]{3in}{Dr. Anupam Joshi, Professor\\
Department of Computer Science\\ 
and Electrical Engineering\\}
\end{singlespace}

Log files comprise of different events happening in various applications, operating systems and even in network devices. Originally they were used to record information for diagnostic and debugging purposes. Nowadays, logs are also used to track events which can be used in auditing and forensics in case of malicious activities or systems attacks. Various softwares like intrusion detection systems, webservers, anti-virus and anti-malware systems, firewalls and network devices generate logs with useful information, that can be used to protect against such system attacks. Analyzing log files can help in proactively avoiding attacks against the systems. The challenging part is understanding log files from unknown devices and of unknown formats.

In this work we are addressing this particular problem. We propose a framework that takes any log file and automatically gives out a semantic interpretation as Linked Open Data in RDF and OWL formats. The framework splits the log files into columns using regular expression-based or dictionary-based classifiers. To classify the columns we use general purpose knowledge bases such as DBpedia and even domain specific ontologies like Cybersecurity and Intrusion detection ontologies[]. Using previous work on inferring semantics of table data[], we capture the relationships between the various columns in a log file. Converting large and verbose log files into such semantic representations will help in better search, integration and rich reasoning over the data.

\par\vfil

